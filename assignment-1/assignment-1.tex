\def\Module{Principles of Computer System Design}
\def\Uebung{Assignment 1}
\def\Studentenname{Marcus Voss (qcz284), R. Schmidtke (rxt809), Marco Eilers (dbk726)}
\def\Sub_date{04.12.2012}

\documentclass[12pt,a4paper]{article}

\usepackage[utf8]{inputenc}
\usepackage[T1]{fontenc}
\usepackage{fullpage} 
\headsep1cm
\parindent0cm
\usepackage{amssymb, amstext, amsmath}
\usepackage{fancyhdr}
\usepackage{lastpage}
\usepackage{booktabs}
\usepackage{graphicx}
\usepackage{subfigure}
\usepackage{hyperref}

\lhead{\textbf{\Module}}
\rhead{\Uebung~(Submission: \Sub_date)}

\cfoot{}
\lfoot{\Studentenname}
\rfoot{\thepage\ of \pageref{LastPage}}
\pagestyle{fancy}
\renewcommand{\footrulewidth}{0.4pt}

\newcommand{\code}[1]{{\fontfamily{fvm}\small \selectfont #1}}

%Line spacing between paragraphs
\setlength{\parskip}{6pt}

\begin{document}

\section*{Exercises} 
\label{sec:exercises}

\subsection*{Question 1}
\label{sec:eq1}

\subsection*{Question 2}
\label{sec:eq2}

\begin{table}[htbp]
\caption{Comparison of Storage}
\begin{center}
\begin{tabular}{|l|l|l|l|}
\hline
 & Hard disk & SSD & Main Memory \\ \hline
1 Access time &  &  &  \\ \hline
2 Average Capacity &  &  &  \\ \hline
3 Cost per unit &  &  &  \\ \hline
4 Reliability &  &  &  \\ \hline
5 Power Consumption &  &  &  \\ \hline
\end{tabular}
\end{center}
\label{tab:storage}
\end{table}

\subsection*{Question 3}
\label{sec:eq3}

\subsection*{Question 4}
\label{sec:eq4}

\section*{Programming Task}
\label{sec:programming}

\subsection*{Question 1}
\label{sec:pq1}
There are basically three types of semantics: 'Exactly once', 'at most once' and 'at least once'. Since all of these three types rely on the interaction of client and server, it depends mostly on the client what type of semantics is employed (as it may choose to try again and again). The service and client implementations we provide are of the type 'at most once'. If the service is unavailable at the time of the request, the request is not repeated, but an exception is raised. Similarly, exceptions are raised on other errors during execution on the server (as per the \code{KeyValueBase} interface). The clients we ship respect that and do not try again, so the overall semantics used are of type 'at most once'. Other client implementations may prefer to retry the operation on failure until it has succeeded (at least) once which can be implemented just fine by keeping track of errors from the service.

\subsection*{Question 2}
\label{sec:pq2}

\subsection*{Question 3}
\label{sec:pq3}
To continuously and consistently test our implementation we created a set of unit tests:
\begin{itemize}
\item SimpleReadWriteTest(N): This test simulates only one client that first inserts N values into the key-value store, and does then N sequential random updates and reads. For every read it checks if the values corresponds to a value that is kept locally in a Java Hash Map. The test fails if any read deviates from the expected value.
\item MultiReadTest(N, h, n): This test simulates h clients (threads). First N values are (sequentially) inserted into the key-value store. Then the h clients make n random reads. For every read each thread checks if the values corresponds to a value that is kept in a shared Java Hash Map. The test fails if any read deviates from the expected value.
\item MultiReadWriteTest(N, h, n) This test is similar to MultiReadTest(), but it will also do n random writes and after each write 10 random reads. As above the test fails if any read deviates from the expected value.
\item AtomicUpdateTest()
\item AtomicScanTest()
\end{itemize}


\subsection*{Question 4}
\label{sec:pq4}

\subsection*{Question 5}
\label{sec:pq5}

data set used is\footnote{Downloaded from: \url{http://an.kaist.ac.kr/~haewoon/release/twitter_social_graph/} [last accessed on: November 30, 2012]}

\subsection*{Question 6}
\label{sec:pq6}

\subsection*{Question 7}
\label{sec:pq7} 

\subsection*{Question 8}
\label{sec:pq8}

\subsection*{Question 9}
\label{sec:pq9}
Kaniner!

\end{document}
