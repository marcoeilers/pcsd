\def\Module{Principles of Computer System Design}
\def\Uebung{Assignment 2}
\def\Studentenname{Marco Eilers (dbk726)}
\def\Sub_date{13.12.2012}

\documentclass[12pt,a4paper]{article}

\usepackage[utf8]{inputenc}
\usepackage[T1]{fontenc}
\usepackage{fullpage} 
\headsep1cm
\parindent0cm
\usepackage{amssymb, amstext, amsmath}
\usepackage{fancyhdr}
\usepackage{lastpage}
\usepackage{booktabs}
\usepackage{graphicx}
\usepackage{subfigure}
\usepackage{hyperref}
\usepackage{threeparttable}
\usepackage{footnote}
\usepackage{listings}
\makesavenoteenv{tabular}

\lhead{\textbf{\Module}}
\rhead{\Uebung~(Submission: \Sub_date)}

\cfoot{}
\lfoot{\Studentenname}
\rfoot{\thepage\ of \pageref{LastPage}}
\pagestyle{fancy}
\renewcommand{\footrulewidth}{0.4pt}

\newcommand{\code}[1]{{\fontfamily{fvm}\small \selectfont #1}}

%Line spacing between paragraphs
\setlength{\parskip}{6pt}

\begin{document}

\title{\Module\\\Uebung}
\author{\Studentenname}
\maketitle

\section*{Exercises} 
\label{sec:exercises}

\subsection*{Question 1}
\label{sec:eq1}

(a) Draw the precedence graph for each schedule. Are the schedules conflict-serializable? Explain
why or why not.

(b) Could the schedules have been generated by a scheduler using strict two-phase locking (strict
2PL)? If so, show it by injecting read/write lock operations in accordance to strict 2PL rules. If
not, explain why.


\subsection*{Question 2}
\label{sec:eq2}

Consider the following scenarios, which illustrate the
execution of three transactions under the Kung-Robinson optimistic concurrency control model. In each
scenario, transactions T1 and T2 have successfully committed, and the concurrency control mechanism
needs to determine a decision for transaction T3. The read and write sets (RS and WS, respectively) for
each transaction list the identifiers of the objects accessed by the transaction.

For each scenario, predict whether T3 is allowed to commit. If T3 is allowed to commit, state which
validation tests were necessary to reach that conclusion. If T3 must be rolled back, state all tests which T3
fails, along with the offending objects from the read and write sets of T1 and T2.


\subsection*{Question 3}
\label{sec:eq3}

(a) In a system implementing force and no-steal, is it necessary to implement a scheme for redo?
What about a scheme for undo? Explain why.

(b) What is the difference between nonvolatile and stable storage? What types of failures are
survived by each type of storage?

(c) In a system that implements Write-Ahead Logging, which are the two situations in which the log
tail must be forced to stable storage? Explain why log forces are necessary in these situations and
argue why they are sufficient for durability.


\subsection*{Question 4}
\label{sec:eq4}

Consider the recovery scenario described in the following, in which we use the
ARIES recovery algorithm. At the beginning of time, there are no transactions active in the system and no
dirty pages. A checkpoint is taken. After that, three transactions, T1, T2, and T3, enter the system and
perform various operations. The detailed log follows:

(a) Apply the ARIES recovery algorithm to the scenario above. Show:
1. the state of the transaction and dirty page tables after the analysis phase;
2. the sets of winner and loser transactions;
3. the values for the LSNs where the redo phase starts and where the undo phase ends;
4. the set of log records that may cause pages to be rewritten during the redo phase;
5. the set of log records undone during the undo phase;
6. the contents of the log after the recovery procedure completes.


\section*{Programming} 
\label{sec:programming}

\subsection*{Question 1}
\label{sec:pq1}

Briefly describe your implementation of the Logger and the Checkpointer in
KeyValueBase. In particular, explain how you achieved overlapping of log I/O and how you
implemented the necessary synchronization for quiescence and checkpointing. (2 paragraphs)


\subsection*{Question 2}
\label{sec:pq2}
Describe how you tested your implementation of both components and ensured that
durability was actually achieved in face of fail-stop failures. (2 paragraphs)


\subsection*{Question 4}
\label{sec:pq4}
Explain how you can quantify the overhead of log-based durability in both throughput and
latency. Design and describe your experimental setup to quantify this overhead. Recall that you must
document all parameters that may influence the performance of the system (e.g., e.g., number of clients,
hardware characteristics, size of dataset managed, mix of operations).


\subsection*{Question 5}
\label{sec:pq5}
Carry out the experiment you described in Question 4 and report your overhead results.
Explain the effects you observe regarding throughput and latency of your service.


\subsection*{Question 6}
\label{sec:pq6}

Briefly describe how you implemented group commit in KeyValueBase. (1 paragraph)



 
\begin{thebibliography}{1}


\end{thebibliography}



\end{document}
