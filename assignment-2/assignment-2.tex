\def\Module{Principles of Computer System Design}
\def\Uebung{Assignment 2}
\def\Studentenname{Marco Eilers (dbk726)}
\def\Sub_date{13.12.2012}

\documentclass[12pt,a4paper]{article}

\usepackage[utf8]{inputenc}
\usepackage[T1]{fontenc}
\usepackage{fullpage} 
\headsep1cm
\parindent0cm
\usepackage{amssymb, amstext, amsmath}
\usepackage{fancyhdr}
\usepackage{lastpage}
\usepackage{booktabs}
\usepackage{graphicx}
\usepackage{subfigure}
\usepackage{hyperref}
\usepackage{threeparttable}
\usepackage{footnote}
\usepackage{listings}
\makesavenoteenv{tabular}

\lhead{\textbf{\Module}}
\rhead{\Uebung~(Submission: \Sub_date)}

\cfoot{}
\lfoot{\Studentenname}
\rfoot{\thepage\ of \pageref{LastPage}}
\pagestyle{fancy}
\renewcommand{\footrulewidth}{0.4pt}

\newcommand{\code}[1]{{\fontfamily{fvm}\small \selectfont #1}}

%Line spacing between paragraphs
\setlength{\parskip}{6pt}

\begin{document}

\title{\Module\\\Uebung}
\author{\Studentenname}
\maketitle

\section*{Exercises} 
\label{sec:exercises}

\subsection*{Question 1}
\label{sec:eq1}

(a) Draw the precedence graph for each schedule. Are the schedules conflict-serializable? Explain
why or why not.

(b) Could the schedules have been generated by a scheduler using strict two-phase locking (strict
2PL)? If so, show it by injecting read/write lock operations in accordance to strict 2PL rules. If
not, explain why.


\subsection*{Question 2}
\label{sec:eq2}

Consider the following scenarios, which illustrate the
execution of three transactions under the Kung-Robinson optimistic concurrency control model. In each
scenario, transactions T1 and T2 have successfully committed, and the concurrency control mechanism
needs to determine a decision for transaction T3. The read and write sets (RS and WS, respectively) for
each transaction list the identifiers of the objects accessed by the transaction.

For each scenario, predict whether T3 is allowed to commit. If T3 is allowed to commit, state which
validation tests were necessary to reach that conclusion. If T3 must be rolled back, state all tests which T3
fails, along with the offending objects from the read and write sets of T1 and T2.


\subsection*{Question 3}
\label{sec:eq3}

(a) In a system implementing force and no-steal, is it necessary to implement a scheme for redo?
What about a scheme for undo? Explain why.

(b) What is the difference between nonvolatile and stable storage? What types of failures are
survived by each type of storage?

(c) In a system that implements Write-Ahead Logging, which are the two situations in which the log
tail must be forced to stable storage? Explain why log forces are necessary in these situations and
argue why they are sufficient for durability.


\subsection*{Question 4}
\label{sec:eq4}

Consider the recovery scenario described in the following, in which we use the
ARIES recovery algorithm. At the beginning of time, there are no transactions active in the system and no
dirty pages. A checkpoint is taken. After that, three transactions, T1, T2, and T3, enter the system and
perform various operations. The detailed log follows:

(a) Apply the ARIES recovery algorithm to the scenario above. Show:
1. the state of the transaction and dirty page tables after the analysis phase;
2. the sets of winner and loser transactions;
3. the values for the LSNs where the redo phase starts and where the undo phase ends;
4. the set of log records that may cause pages to be rewritten during the redo phase;
5. the set of log records undone during the undo phase;
6. the contents of the log after the recovery procedure completes.


\section*{Programming} 
\label{sec:programming}

\subsection*{Question 1}
\label{sec:pq1}

Both Logger and Checkpointer are threads that are started once at the beginning and continue to run as long as the web service is available. The Logger has a queue of log requests, which consist of a LogRecord and a Future. For each call of \texttt{logRequest}, a new Future is created. This future, along with the incoming LogRecord, is then put in said queue and the Future is returned immediately. The Logger thread polls the queue in an endless loop, and for every pair of LogRecord and Future it first writes (and forces) the LogRecord to disk (by simply serializing the LogRecord object) and then sets the Future's status to done. Any methods that manipulate the database therefore have to call logRequest and then wait until the returned Future's state is changed to done.

The Checkpointer is also a thread running in an endless loop. It waits for a certain amount of time and then calls \texttt{quiesce} on KeyValueBaseLogImpl, another object that is created only once at the beginning. Quiesce is implemented via a MasterLock, another ReentrantReadWriteLock. All other operations now request a shared ("read") lock from this central component before they do anything else and return the lock when everything is done, which means that those operations can still run concurrently. \texttt{quiesce}, however, requests an exclusive ("write") lock, which means that as soon as this lock is acquired, all other operations can no longer get their read lock and have to wait. It is guaranteed that the function will get its write lock at some point (although it may take a while), since I use the fair scheduler. \texttt{resume} then simply releases the write lock again. Between those two function calls, the Checkpointer flushes the Store, writes the current Index to disk and truncates the log. If the key-value-store is currently initialized, it will also write an empty init operation (meaning a call to init that does not point to an initialization file) in the new log, so that during recovery the store knows whether it has already been initialized.

When the server is started and it detects that a log file is present, it first deserializes the Index, updates it to use the MemoryMappedFile that is on disk, and then steps through the log file and re-invokes every method call that is recorded in it. A flag is set before this, so that REDOing these operations is not logged again.


\subsection*{Question 2}
\label{sec:pq2}
Describe how you tested your implementation of both components and ensured that
durability was actually achieved in face of fail-stop failures. (2 paragraphs)


\subsection*{Question 4}
\label{sec:pq4}
Explain how you can quantify the overhead of log-based durability in both throughput and
latency. Design and describe your experimental setup to quantify this overhead. Recall that you must
document all parameters that may influence the performance of the system (e.g., e.g., number of clients,
hardware characteristics, size of dataset managed, mix of operations).


\subsection*{Question 5}
\label{sec:pq5}
Carry out the experiment you described in Question 4 and report your overhead results.
Explain the effects you observe regarding throughput and latency of your service.


\subsection*{Question 6}
\label{sec:pq6}

Briefly describe how you implemented group commit in KeyValueBase. (1 paragraph)



 
\begin{thebibliography}{1}


\end{thebibliography}



\end{document}
